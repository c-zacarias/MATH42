\documentclass[notitlepage]{article}

\usepackage{amsmath}
\usepackage{amsthm}
\usepackage{amssymb}
\usepackage{enumerate}
\usepackage{fancyhdr}
\usepackage{blindtext}

\theoremstyle{plain}
\newtheorem{thm}{Theorem}[section] % reset theorem numbering for each section

\theoremstyle{definition}
\newtheorem{defn}[thm]{Definition} % definition numbers are dependent on theorem numbers
\newtheorem{exmp}[thm]{Example} % same for example numbers

\begin{document}

\begin{titlepage}

\title{Discrete Mathematics Notes}
\author{Charlie Zacarias}
\date{Monday, August 28, 2017}
\maketitle
\renewcommand{\section}[2]{}

\noindent\textbf{\large{Contents}}

\tableofcontents

\end{titlepage}

\setcounter{page}{2}

\pagestyle{fancy}
\lhead{Discrete Mathematics Notes}
\rhead{Aug. 28, 2017}
\lfoot{MATH 42}
\cfoot{\thepage}
\rfoot{So, W.}
\renewcommand{\headrulewidth}{0.4pt}
\renewcommand{\footrulewidth}{0.4pt}

\section{Propositions}

  \defn{A \textbf{proposition} is a declarative statement that is either
  true or false, but not both nor none.}

  True (T) and False (F) denote \textbf{truth values} such that propositions
  either have a true value (T) or a false value (F).

  \exmp{Here are some examples of some simple numerical propositions:
  \begin{enumerate}[(i)]
    \item $1+2=3$ has a truth value T because the result, or \textbf{conclusion}
    holds.
    \item $2+3=4$ has a truth value F because the conclusion does not hold for
    the given assumption, or \textbf{condition}.
    \item $71951 = 2*17*59*239$ has a truth value F  because the conclusion does
    not hold for the given condition (the $2$ on the right-hand side makes the
    product even, whereas $71951$ is odd).
    \item $4^{101} -1$ is divisible by $3$. \\
    We will show the truth value of this statement by performing the following
     operations: $4^3 -1 = a^3 - b^3 = (a-b)(a^2 + ab + b^2)$. If $(a-b) \implies
     (4-1)=3$, then $4^3 - 1$ is divisible by 3. Therefore $4^{101} -1$ is divisible
     by $3$ as well.
  \end{enumerate}}

  \exmp{Here are examples where we introduce \textbf{compound propositions}, or
  new propositions formed by logical operators. More information on why these
  compound propositions are true will be explained in the next section. Assume
  $a,b \in \mathbb{R}$:
  \begin{enumerate}[(i)]
    \item $2*3 = 3*2$ and $2+3 = 3+2$ holds true (T) because if both propositions
     are true, then the compound proposition must be true.
    \item If $a > 3$, then $2a > 6$ is true (T) because if $a$ is always greater
    than $3$, then the conclusion must always hold true.
    \item $a \leq 3$ or $2a > 6$ is true (T) since the truth value depends on
    whether $a$ is $3$ or less than $3$, thus we assume it is true.
    \item If $a+b=0$ then $a=0$ or $b=0$ is false (F) because $a$ and $b$ both
    have to be $0$ in order for the if-statement to be true.
    \item $a*b=0$ if and only if $a=0$ or $b=0$ is true (T) because only $a$ or $b$ has
    to equal $0$ in order for the proposition to be true.
  \end{enumerate}

\section{Logical Operators}
asdfghjkl;

\newpage
fasdfsdfasdfsdaf

\end{document}
